\documentclass{beamer}
%
% Choose how your presentation looks.
%
% For more themes, color themes and font themes, see:
% http://deic.uab.es/~iblanes/beamer_gallery/index_by_theme.html
%
\mode<presentation>
{
  \usetheme{default}      % or try Darmstadt, Madrid, Warsaw, ...
  \usecolortheme{default} % or try albatross, beaver, crane, ...
  \usefonttheme{default}  % or try serif, structurebold, ...
  \setbeamertemplate{navigation symbols}{}
  \setbeamertemplate{caption}[numbered]
} 

\usepackage[english]{babel}
\usepackage[utf8x]{inputenc}


\title[Your Short Title]{Context-Aware-VR}
\author{Charles Derek}
\institute{University of Antwerp}
\date{7 september 2020}

\begin{document}

\begin{frame}
  \titlepage
\end{frame}

% Uncomment these lines for an automatically generated outline.
%\begin{frame}{Outline}
%  \tableofcontents
%\end{frame}

\section{Introduction}

\begin{frame}{Context extraction}
Context type information: 

\begin{itemize}
	\item Eye tracking
	\item Location tracking 
	\item Rotation of the head
\end{itemize}	
\vspace{0.8cm}
	
$\rightarrow$ Mobility\ pattern

\vspace{1.23cm}

$\rightarrow$ Basis\ for\ criteria\ to\ choose\ appropriate\ VR\ headsets
\end{frame}


\begin{frame}{VR headset survey}
	\begin{figure}
		\includegraphics[scale=0.36]{"vr-one-plus".jpg}
		\caption{Phone-driven VR}
	\end{figure}
	
	Advantages:
	\begin{itemize}
		\item Built-in gyroscope $\rightarrow$ rotation
		\item Location $\rightarrow$ through phone or indoor positioning system (Ken)
	\end{itemize}
	
	Disadvantages:
	\begin{itemize}
		\item Eye tracking difficult $\rightarrow$ position of front facing camera
		\item Precision $\rightarrow$ dependent on phone
		\item Everything needs to be app-driven $\rightarrow$ complexity increase
	\end{itemize}
\end{frame}


\begin{frame}{VR headset survey}
	\begin{figure}
		\includegraphics[scale=0.06]{"vive-pro".jpg}
		\caption{HTC Vive Pro Eye}
	\end{figure}
	
	Advantages:
	\begin{itemize}
		\item Rotation $\rightarrow$
		\item Location $\rightarrow$ 
		\item Eye tracking $\rightarrow$ tobii xr sdk or Vive SRanipal SDK
	\end{itemize}
	
	Disadvantages:
	\begin{itemize}
		\item Unity only
		\item Precision $\rightarrow$ dependent on phone
		\item Everything needs to be app-driven $\rightarrow$ complexity increase
	\end{itemize}
\end{frame}



\begin{frame}{VR headset survey}
	\begin{figure}
		\includegraphics[scale=0.28]{"tobii-htc-vive".png}
		\caption{Tobii HTC VIVE Devkit/Tobii Pro VR Integration}
	\end{figure}
	
	Advantages:
	\begin{itemize}
		\item Supports: Unity, Unreal and native 
		\item 
	\end{itemize}
	
	Disadvantages:
	\begin{itemize}
		\item 
	\end{itemize}
\end{frame}


\begin{frame}{VR headset survey}
	\begin{figure}
		\includegraphics[scale=0.6]{"varjo".jpeg}
		\caption{Varjo v2}
	\end{figure}
	

	\begin{itemize}
		\item Business only
		\item VR/AR
		\item €5000 and more $\rightarrow$ software
	\end{itemize}
\end{frame}


%https://www.engadget.com/2019-09-24-7invensun-droolon-f1-eye-tracking-vr-htc-vive-headsets.html?guccounter=1&guce_referrer=aHR0cHM6Ly93d3cuZ29vZ2xlLmNvbS8&guce_referrer_sig=AQAAAAijwv7BnA_gwAYzYt13CTz4YLgRuP4xGGJWHkW87vjQiFMYLIB2Ya1i7VS8l4XAO4vCaR0JW55MmiKBSNpJsLIz3LZOWjZJBvwieZBaV6EWMsywfPLVi_R4BDVP5QnxsdZE97MHHuSEfELMXM390qtq47UmEGSNY912sDZxgVfx

\begin{frame}{VR headset survey}
	VR headsets with eye-tracking as a module:
	\vspace{0.5cm}
	
	PupilLabs Binocular Add-on:
	\begin{itemize}
		\item €1400!
		\item High compatibility: $\rigtharrow$ Python, Unity,...
		\item HTC Vive, Vive PRO or Vive Cosmos VR
    \end{itemize} 
	
\begin{figure}
	\centering
	\subfloat{{\includegraphics[width=4cm]{"pupillabs".png} }}%
	\qquad
	\subfloat{{\includegraphics[width=4.5cm]{"droolon".jpg} }}%
	\caption{Pupillabs vs. Droolon f1}%
	\label{fig:example}%
\end{figure}

	Droolon f1:
	\begin{itemize}
		\item Only costs around €150
		\item Vive Cosmos, Vive Focus Plus, Vive Focus or the original Vive aka Vive CE
	\end{itemize}
\end{frame}

\end{document}
